\chapter{Анализ предметной области}

\section{Шифрование данных}

История шифрования началась с древних времён, когда ещё не было
компьютеров и других современных вычислительных средств. Наши предки
использовали шифрование только для замены символов. Процесс расшифровки был долгим и утомительным.

В прежние десятилетия, когда у людей не было проблем с защитой
информации своих компьютеризированных систем, шифрование
использовалось только государственными органами для облегчения передачи
секретной информации во время их общения. В настоящее время шифрование
широко применяется для передачи данных через сети, устройства Bluetooth, банкоматы и так далее~\cite{encrypting}.

\section{Симметричное шифрование данных}

Симметричное шифрование, также известное как шифрование с
секретным ключом, выполняется с использованием одного и того же ключа
для шифрования и расшифровки информации. Ключ также
отправляется получателю вместе с секретным текстом, и с помощью этого
ключа выполняется преобразование зашифрованного текста в обычный~\cite{sim-encrypting}.

\section{Асимметричное шифрование данных}

В асимметричном шифровании, также известном как шифрование с
открытым ключом, есть 2 разных ключа -- открытый и закрытый. Ключи расшифровки и шифрования отличаются друг от друга. Ключ к шифру -- это открытый ключ, а дешифратор -- закрытый ключ. Текст, зашифрованный с помощью открытого ключа, может быть расшифрован только с помощью закрытого ключа, принадлежащего этому пользователю. Это делает асимметричный алгоритм более эффективным, чем симметричное шифрование~\cite{asim-encrypting}.

\section{Надёжность асимметричного шифрования}

Теоретически приватный ключ от асимметричного шифра можно вычислить, зная публичный ключ и механизм, лежащий в основе алгоритма шифрования (последнее -- открытая информация). Надёжными считаются шифры, для которых это нецелесообразно с практической точки зрения. Так, на взлом шифра, выполненного с помощью алгоритма RSA с ключом длиной 768 бит на компьютере с одноядерным процессором AMD Opteron с частотой 2,2 ГГц, бывшем в ходу в середине 2000-х, ушло бы 2000 лет.

При этом фактическая надёжность шифрования зависит в основном от длины ключа и сложности решения задачи, лежащей в основе алгоритма шифрования, для существующих технологий. Поскольку производительность вычислительных машин постоянно растёт, длину ключей необходимо время от времени увеличивать. Так, в 1977-м (год публикации алгоритма RSA) невозможной с практической точки зрения считалась расшифровка сообщения, закодированного с помощью ключа длиной 426 бит, а сейчас для шифрования этим методом используются ключи от 1024 до 4096 бит, причём первые уже переходят в категорию ненадёжных.

Что касается эффективности поиска ключа, то она незначительно меняется с течением времени, но может скачкообразно увеличиться с появлением кардинально новых технологий (например, квантовых компьютеров). В этом случае может потребоваться поиск альтернативных подходов к шифрованию~\cite{asim-kasp}.
