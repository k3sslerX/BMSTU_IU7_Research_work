\ssr{ВВЕДЕНИЕ}

Шифрование данных  -- ключевая технология в области информационной безопасности, обеспечивающая защиту конфиденциальной информации в процессе её передачи и хранения. В условиях быстрого развития технологий и повсеместного использования интернета, вопросы безопасности данных становятся всё более актуальными. 

Важность шифрования невозможно переоценить, так как оно является основой для создания доверительных и защищённых коммуникационных каналов, а также для предотвращения утечек данных и кибератак.

Одним из вариантов шифрования данных является асимметричное шифрование данных.

Асимметричное шифрование данных, также известное как криптография с открытым ключом, представляет собой одну из самых мощных и безопасных технологий защиты информации. В отличие от симметричного шифрования, где для шифрования и расшифровки данных используется один и тот же ключ, в асимметричной криптографии применяется пара ключей: открытый и закрытый. Открытый ключ используется для шифрования данных, а закрытый — для их расшифровки. Эта схема гарантирует высокий уровень безопасности, так как даже если злоумышленник получит доступ к открытому ключу, он не сможет расшифровать информацию без закрытого ключа, который хранится в секрете у владельца.~\cite{asim-kasp}

Цель работы -- проанализировать существующие методы асимметричного шифрования данных.

Для достижения цели необходимо выполнить следующие задачи:
\begin{enumerate}
	\item ознакомиться с существующими алгоритмами асимметричного шифрования данных;
	\item выделить их сходства и различия;
	\item сформулировать критерии для сравнения алгоритмов;
	\item провести сравнительный анализ алгоритмов по сформулированным критериям.
\end{enumerate}